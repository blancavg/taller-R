\chapter{Introducci'on}

%-----------------------------------------------------------
\section{Instalaci'on}
\begin{itemize}
    \item Instalar R~\cite{Rproject} . Disponible para Linux, MacOS X y Windows:\\
 en \url{http://cran.r-project.org/mirrors.html}.
    \item Puede usarse en consola o instalar un IDE.\\
Sugerencia: RStudio \url{http://www.rstudio.org/}.
\end{itemize}

El conjunto de datos que se usar'a es credit-g.csv~\cite{credit:2010}.

En modo consola:
\begin{verbatim}
$ R
\end{verbatim}

%-----------------------------------------------------------
\section{C'omo obtener ayuda}
\begin{verbatim}
help.start()        # ayuda general
help(nombrefuncion) # detalles sobre la funcion
?funcion            # igual que el anterior
apropos("solve")    # lista las funciones que contienen "solve"
example(solve)      # muestra un ejemplo del uso de solve
help("*")
vignette()           
vignette("foo")     
data()              # muestra los conjuntos de datos disponibles
help(datasetname)   # detalles del conjunto de datos
\end{verbatim}

%-----------------------------------------------------------
\section{Espacio de trabajo}
Es el entorno de tu sesi'on actual en R e incluye cualquier objeto: vectores, matrices, dataframes, listas, funciones. Al finalizar la sesi'on se puede guardar el actual espacio de trabajo para que autom'aticamente se cargue en la siguiente.

Algunos comandos est'andar para definir el espacio de trabajo son los siguientes:

\begin{verbatim}
getwd() # muestra el directorio actual
ls()    # lista los objetos en el espacio de trabajo
setwd(mydirectory)      # cambia el path a mydirectory
setwd("c:/docs/mydir")  # notar / en vez de \ en Windows
setwd("/usr/rob/mydir") # en Linuz
history() # despliega los 25 comandos recientes
history(max.show=Inf) # despliega los comandos previos
q() # quit R. 
\end{verbatim}

\section{Packages}
\begin{verbatim}
.libPaths() # obtiene ubicación de la librería
library()   # muestra los paquetes instalados
search()    # muestra los paquetes cargados
# descarga e instala paquetes del repositorio CRAN
install.packages("nombredelpaquete")  
library(package) # carga el paquete
\end{verbatim}

\section{Scripts}
\begin{verbatim}
# en Linux
R CMD BATCH [options] my_script.R [outfile] 

# en ms windows (ajustar el path a R.exe)
"C:\Program Files\R\R-2.5.0\bin\R.exe" CMD BATCH
   --vanilla --slave "c:\my projects\my_script.R" 

source("myfile")
sink("record.lis") # direcciona la salida al archivo record.lis
\end{verbatim}

%http://cran.r-project.org/doc/manuals/R-intro.html#Data-permanency-and-removing-objects
%http://www.statmethods.net/input/index.html
%http://www.cyclismo.org/tutorial/R/input.html
%-----------------------------------------------------------


