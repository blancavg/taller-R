\chapter{Tipos de datos y funciones b'asicas}

%-----------------------------------------------------------
\section{N'umeros}

\begin{verbatim}
var1 <- 54
var1

var2 <- sqrt(var1*8)
var2
\end{verbatim}
%-----------------------------------------------------------
\section{Vectores}

\begin{verbatim}
vector <- c(1,2,3,4,5)
vector[0]
vector[1]

cadena <- "uno"
cadena

lcadena <- c("casa","manzana","uva")
lcadena

vlogico <- c(TRUE,FALSE,TRUE,TRUE,FALSE)
vlogico
\end{verbatim}

%-----------------------------------------------------------
\section{Dataframes}

\begin{verbatim}
c1 <- c(25,26,27,28)
c2 <- c("Ana", "Lola", "Luis", "Pedro")
c3 <- c(TRUE,TRUE,TRUE,FALSE)
mydata <- data.frame(c1,c2,c3)
names(mydata) <- c("ID","Nombre","Aprobado") # nombres de variables
\end{verbatim}

Otros tipos de datos son: arrays, listas y factores.
%-----------------------------------------------------------
\section{Exportar e importar datos}
\begin{verbatim}
    > setwd("path_name")
    > credit <- read.csv(file = "../data/credit-g.csv", sep = ",",
                         na.strings = "NULL")
\end{verbatim}
%-----------------------------------------------------------
\section{Despliegue de objetos}
\begin{verbatim}
ls()  # lista de objetos
names(credit) # variables de credit
str(credit) # estructura de credit
levels(credit$foreign_worker) # niveles(valores de variable)
dim(credit) # dimensiones (ren x cols)
class(credit) # clase del objeto
credit  # objeto
head(credit, n=10)
tail(credit, n=10)
credit$purpose
#purpose
attach(credit)  # coloca la bd en el path
purpose
\end{verbatim}
%-----------------------------------------------------------
\section{Renglones, columnas, m'inimo y m'aximo}

\begin{verbatim}
# Explorando conteo
ren <- nrow(credit)   # raw row count
col <- ncol(credit)

# Min y Max 
agemin <- min(age,na.rm = TRUE) # min sin NULLS en age
agemax <- max(age,na.rm = TRUE)

ammin <- min(credit_amount,na.rm = TRUE) # min sin NULLS en age
ammax <- max(credit_amount,na.rm = TRUE)
\end{verbatim}
%-----------------------------------------------------------
\section{Subconjuntos}

\begin{verbatim}
# Filtrando por edad
mayor30 <- subset(credit, age >= 30)
renmayor30 <- nrow(mayor30)
renmayor30

en20y40 <- subset(credit, age >= 20  & age <= 40)
ren20y40 <- nrow(en20y40)
ren20y40
\end{verbatim}

%-----------------------------------------------------------
\subsection{Muestra aleatoria}

\begin{verbatim}
set.seed(32) 
dsmall <- credit[sample(nrow(credit), 10), ]
dsmall
\end{verbatim}

%-----------------------------------------------------------
\subsection{Definiendo rangos}
\begin{verbatim}
# Filtrando por bins 
agebin = cut(age,breaks = c(18,30,40,50,60,70,80))
agebinfile <- data.frame(purpose,credit_amount,personal_status,
              housing,job,age=agebin,class)
agebinfile
\end{verbatim}

%-----------------------------------------------------------
\section{'Unicos, frecuencia, orden}
\begin{verbatim}
# unicos
unicos <- unique(class)
unicos
nhousing <- length(unique(class))
nhousing
\end{verbatim}

\begin{verbatim}
# table
mycredit <- table(dsmall$class)
mycredit
mycredit <- table(dsmall$age,dsmall$credit_amount)
mycredit
\end{verbatim}

\begin{verbatim}
# ordenar
mycredit <- table(dsmall$housing)
mycredit
mylist <- sort(mycredit,decreasing = TRUE)
mylist
\end{verbatim}

%-----------------------------------------------------------
\section{Construyendo archivos de salida}

\begin{verbatim}
oframe = data.frame(purpose,credit_amount,personal_status,
                    housing,job,age=agebin,class)
write.table(oframe, row.names = FALSE, 
            sep = ";",quote = FALSE, file="../data/output01_data.csv")
\end{verbatim}
