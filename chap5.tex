\chapter{Miner'ia de datos}

La miner'ia de datos ha atra'ido gran inter'es debido a la cada vez mayor disponibilidad de gran cantidad de datos y la necesidad de transformarlos en informaci'on y conocimiento 'util. R cuenta con librer'ias que facilitan la aplicaci'on de t'ecnicas de miner'ia.\\
\url{http://www.the-data-mine.com/bin/view/Software/R}

%-----------------------------------------------------------
\section{Clasificaci'on}

Las enormes cantidades de datos acumuladas en bases de datos pueden ser usadas para tomar decisiones. La clasificaci'on y la predicci'on son dos formas de an'alisis de datos que pueden usarse para extraer modelos que describen clases o predicen tendencias futuras de los datos. Mientras que la clasificaci'on predice etiquetas o variables categ'oricas, o valores discretos, los modelos predictivos lo hacen con funciones cont'inuas. Por ejemplo, un modelo de clasificaci'on puede construirse para categorizar si un pr'estamo bancario es seguro o riesgoso, mientras que un modelo de predicci'on puede construirse para predecir gastos de clientes potenciales en equipo de c'omputo dado su ingreso y ocupaci'on. \\

Regresi'on lineal \\
En la regresi'on lineal, los datos se modelan usando una l'inea recta. La regresi'on lineal es la forma de regresi'on m'as simple. La regresi'on de dos variables modela una variable, $Y$, la variable de salida como funci'on lineal de otra variable aleatoria, $X$, el predictor, i.e., $Y = \alpha +  \beta X$\\

\url{http://www.youtube.com/watch?v=ZGTBOhbahmY}\\
\url{http://msenux.redwoods.edu/math/R/regression.php}\\
\url{http://scc.stat.ucla.edu/page_attachments/0000/0139/reg_1.pdf}\\

Un 'arbol de decisi'on es como un diagrama de flujo con estructura de 'arbol en el que cada nodo interno denota una prueba en un atributo. Cada rama representa la salida de una prueba, y los nodos hoja representan clases. Para clasificar un ejemplo desconocido, los valores de atributo se prueban contra el 'arbol. Se traza un camido desde la ra'iz hasta el nodo hoja que indica la clase. Los 'arboles de decisi'on pueden convertirse f'acilmente a reglas de clasificaci'on.\\
Arboles \url{http://en.wikibooks.org/wiki/Data_Mining_Algorithms_In_R/Classification/Decision_Trees}\\
\url{http://hisdu.sph.uq.edu.au/lsu/adrian/treecode.htm}\\
\url{http://www.statmethods.net/advstats/cart.html}\\

KNN \url{http://en.wikibooks.org/wiki/Data_Mining_Algorithms_In_R/Classification/kNN}\\
Clasificaci'on por retropropagaci'on. La retropropagaci'on es el algoritmo de aprendizaje de redes neuronales. El campo de las redes neuronales fue acu~nado por psic'ologos y neurobi'ologos. B'asicamente, una red neuronal es un conjunto de unidades de entrada/salida conectadas en el que cada conexi'on tiene un peso asociado. Durante la fase de aprendizaje, la red aprende ajustando los pesos de tal manera que sea capaz de predecir la clase correcta de los ejemplos de entrada.

%-----
\section{Agrupaci'on}
KMeans \url{http://www.statmethods.net/advstats/cluster.html}\\
\url{http://stat.ethz.ch/R-manual/R-devel/library/stats/html/kmeans.html}\\
%-----
\section{Reglas de asociaci'on}

Apriori \url{http://prdeepakbabu.wordpress.com/2010/11/13/market-basket-analysisassociation-rule-mining-using-r-package-arules/}\\
\url{http://lyle.smu.edu/IDA/arules/}\\
\url{http://paginas.fe.up.pt/~ec/files_0506/R/arules.pdf}\\
\url{http://cran.r-project.org/web/packages/arulesViz/vignettes/arulesViz.pdf}\\
\url{http://www1.ccls.columbia.edu/~ansaf/lecture1.pdf}\\
%-----
\section{Selecci'on de atributos}

FSelector package \url{http://en.wikibooks.org/wiki/Data_Mining_Algorithms_In_R/Dimensionality_Reduction/Feature_Selection}\\

Caret package \url{http://cran.r-project.org/web/packages/caret/vignettes/caretSelection.pdf}\\

Boruta \url{http://www.jstatsoft.org/v36/i11/paper}\\
%-----
\section{Miner'ia de textos}
tm package \url{http://cran.r-project.org/web/packages/tm/vignettes/tm.pdf}\\

text mining infraestructure in R \url{http://www.jstatsoft.org/v25/i05/paper}\\

Rkea \url{http://cran.r-project.org/web/packages/RKEA/vignettes/kea.pdf}\\
%-----
\section{Otros tratamientos de los datos}
Machine Learning \url{http://cran.r-project.org/web/views/MachineLearning.html}\\

R with Python \url{http://www2.warwick.ac.uk/fac/sci/moac/students/peter_cock/r/rpy/}\\


Libro:\\
\url{http://www.r-bloggers.com/finally-a-practical-r-book-on-data-mining-data-mining-with-r-learning-with-case-studies-by-luis-torgo/}


\url{http://www.johndcook.com/R_language_for_programmers.html}\\

Data mining with R \url{http://cos.name/wp-content/uploads/2008/12/data-mining-with-r-by-john-maindonald.pdf}\\

Comparison: \url{http://brenocon.com/blog/2009/02/comparison-of-data-analysis-packages-r-matlab-scipy-excel-sas-spss-stata/}\\

Otros tratamientos: \url{http://www.statmethods.net/index.html}\\

